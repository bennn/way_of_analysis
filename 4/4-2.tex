\documentclass{article}
\usepackage{ben}

\vspace*{0.1cm}
\section*{\S 4.2}
\begin{enumerate}
\item[4.2.4.1]
  The function's monotone increasing, so we know that if 
  $f(x_0)$ exists, then $\forall x_1 < x_0, f(x_1) \le f(x_0)$.
  Also, $\forall x_2 > x_0, f(x_2) \ge f(x_0)$.
  So $f(x_1)$ and $f(x_2)$ bound the value of $f(x_0)$;
  it cannot be any less than the former or any greater
  than the latter. For a jump discontinuity at $x_0$, 
  these boundaries translate to bounds on the jump. 
  Hence any jump can start no lower than $f(x_1)$
  and go no higher than $f(x_2)$. In other words, 
  the magnitude of the jump is no more than
  $f(x_2) - f(x_1)$.

\item[4.2.4.2]
  Take successive points
  $(b-\frac{1}{2}, b-\frac{1}{3}, b-\frac{1}{4},\ldots)$.
  We know that mapping $f$ over the sequence produces
  an increasing sequence. What can happen to these
  values? In one case, they don't get closer together as 
  we approach $b$. But they must increase, so they tend
  towards $\infty$ as a limit. 
  Otherwise, they converge to a non-infinte limit.

\item[4.2.4.3]
  If the domain is continuous, we can always find 
  points of the image within an epsilon neighborhood 
  for two points in the preimage that exist within some
  epsilon. So there's no way the image is not an interval.

  The image will be an open interval if the preimage is
  open (for example, the identity function on the real line)
  or if elements of the image get arbitrarilty close to 
  $\infty$ or a point (say zero) without ever passing it.

\item[4.2.4.4]
  
\item[4.2.4.5]

\end{enumerate}
