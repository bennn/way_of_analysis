\documentclass{article}
\usepackage{amsmath, amssymb, fullpage, listings}
\begin{document}
\vspace*{0.1cm}

\section*{\S 3.1}
\begin{enumerate}
\item[3.1.3.1]
  \begin{enumerate}
  \item
    %% 0, 1.5, -.66, 1.25, -.8, ...
    \begin{description}
    \item[sup]
      1.5
    \item[inf]
      -1
    \item[limsup]
      1
    \item[liminf]
      -1
    \item[limits]
      1, -1
    \end{description}
  \item
    \begin{description}
    \item[sup]
      $\frac{3}{2}$
    \item[inf]
      0
    \item[limsup]
      1
    \item[liminf]
      1
    \item[limits]
      1
    \end{description}
  \item
    \begin{description}
    \item[sup]
      2
    \item[inf]
      -3
    \item[limsup]
      1
    \item[liminf]
      -3
    \item[limits]
      -3, 1
    \end{description}
  \end{enumerate}
\item[]
\item[3.1.3.2]
  % No, the sequences don't necessarily converge. 
  % Example: $\{y_n\} = .9, .99, .999, \ldots$ 
  % $\{z_n\} = -.9, -.99, -.999, \ldots$ 
  % Bounding the sequences doesn't help. 
  % My example sequences are bounded. 
  % Yeah, I can't even think of an example 
  % where $\{y_n\}$ and $\{z_n\}$ are
  % \emph{unbounded} but the sequence 
  % $x_n = y_n + z_n$ \emph{is} bounded.
\item[]
\item[3.1.3.3]
  $y_n$ is an upper bound for $E$ because it is the limit of upper bounds. 
  $y_n$ is the least upper bound because it is the limit of elements $x_n$
        of $E$.
\item[]
\item[3.1.3.4]
  The union takes elements from both sets, so the suprenam can only increase. 
  The intersection only removes elements. If $sup(B) > sup(A)$, then we
  know that $sup(B) \not\in A$ and therefore $sup(B) \not\in A \cap B$.
\item[]
\item[3.1.3.5]

  
\item[]
\item[3.1.3.6]
  Yes. 
  %% Every elements of sub-sub-sequence $\{x^{\prime\prime}_n\}$
  %% is an element of subsequence $\{x^{\prime}_n}$; in turn, every element
  %% of $\{x^\prime_n\}$ is an element of $\{x_n\}$. Thus 
  Members of the subsubsequence are members of the parent sequence,
  so one can define a subsequence selection function to extract it from
  the parent sequence. 
 \item[]
\item[3.1.3.7]
  Create two infinite matrices:
\item[]
  \begin{tabular}{ c c c c c}
   0 & 1 & 2 & 3 & \ldots \\
   0 & 1 & 2 & 3 & \ldots \\
   0 & 1 & 2 & 3 & \ldots \\
   0 & 1 & 2 & 3 & \ldots \\
   \vdots & \vdots & \vdots & \\
  \end{tabular}
  \begin{tabular}{ c c c c}
  -1 & -2 & -3 & \ldots \\
  -1 & -2 & -3 & \ldots \\
  -1 & -2 & -3 & \ldots \\
  -1 & -2 & -3 & \ldots \\
   \vdots & \vdots & \vdots & \\
  \end{tabular}

  Diagonialize each and shuffle those resulting sequences. 
  Now we have a sequence in which each integer appears an infinite
  number of times. 
\item[]
\item[3.1.3.8]
  Let $S$ be a sequence having a subsequence $S^\prime$ with limit $+\infty$. 
  By definition, there are an infinite number of elements in $S^\prime$
  which get arbitrarily close to $+\infty$. $S^\prime \subset S$, so $S$
  also has an infinite number of points approximating $+\infty$, so 
  $+\infty$ is a limit of $S$.

  If $+\infty$ is a limit point of a sequence $\{ x_n \}$, then 
  we can obtain a subsequence that has the same limit by deleting
  a finite number of elements from $\{ x_n \}$. 
  
\item[]
\item[3.1.3.9]
%   Yes. Create another matrix and diagonalize it. The liminf is zero.
% \item[]
%   \begin{tabular}{c c c c}
%    1 & $\frac{1}{2}$ & $\frac{1}{3}$ & \ldots \\
%    1 & $\frac{1}{2}$ & $\frac{1}{3}$ & \ldots \\
%    1 & $\frac{1}{2}$ & $\frac{1}{3}$ & \ldots \\
%    1 & $\frac{1}{2}$ & $\frac{1}{3}$ & \ldots \\
%    \vdots & \vdots & \vdots & \\
%   \end{tabular} 
  This is not possible. If there existed a sequence 
  with the set $x_1,\ldots,x_n$ of limit points such
  that $x_i = \frac{1}{i}$, then the liminf of this sequence
  would be $0$. The liminf of the sequence is also a limit point, 
  yet $0$ is not of the form $\frac{1}{i}$.
  
\item[]
\item[3.1.3.10]
  Limits of subsequences are limits of sequences. If an infinite
  number of terms in a subsequence approach a limit point, then it 
  follows that an infinite number of terms in the sequence approach the
  same limit. They're the same terms!

  Sequences have no limits that do not belong to their subsequences. 
  Assume, for contradiction, that the sequence $x_n = y_n + z_n$ has 
  a limit point $l$ such that neither an infinite number of terms 
  of $y_n$ nor an infinite number of terms of $z_n$ approach $l$. 
  In other words, no more than a finite number of $y\in y_n$ and 
  $z \in z_n$ approximate the limit. Then only finitely many $x\in x_n$
  could possibly approach $l$, a contradiction.

  Shuffling doesn't matter. The order in which subsequence terms appear
  in the final sequence has no bearing on whether an infinite number of 
  sequence terms fall within some epsilon of a limit.
\item[]
\item[3.1.3.11]
  Rows and columns appear as subsequences, so their limits appear in the
  sequence. A definition for the row subsequence selection function, 
  defined in terms of the row number, $r$, where the top row of the matrix
  is row $r=1$:

  \[f_r = 1\ \text{when}\ r=0\]

    \[f_r(i) = f_{r-1}(1) + r - 1\ \text{when}\ i=1\]

    \[f_r(i) = f_r(i-1) + i + r - 1\ \text{else}\ \]

  %% What does "all limit points" mean? All row/col limits or all limits 
  %% of matrix entries?
  Yes, one necessarily gets all limit points in this way. 
  All the matrix terms appear in the sequence. How could you not?

\item[]
\item[3.1.3.12]
  Reflexivity \& symmetry are easy. Any sequence differs from 
  itself in finitely many (i.e. zero) terms, and if $x$ differs from 
  $y$ in $n$ terms then obviously $y$ and $x$ will be identical after $n$
  terms. 
  For transitivity, say $x$ and $y$ differ by $n$ terms and $y$ and $z$
  differ by $m$ terms. Then $x$ and $z$ differ by $max(n,m)$ terms. 
  If $z$ is  the same sequence as $y$ after a finite number of steps
  then it's obviously also identical to $x$ after finitely many steps. 

  Equivalent sequences have the same limit points because they have
  the same behavior at infinity. They are exactly the same for 
  an infinite number of terms. 
\end{enumerate}

\end{document}
