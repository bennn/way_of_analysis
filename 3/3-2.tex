\newpage
\vspace*{0.1cm}
\section*{\S 3.2}
\begin{enumerate}
\item[3.2.3.1]
  Let $A=(a, b)$ be the open set and let 
  $x_1, x_2, \ldots x_n$ be the points removed from $A$,
  arranged in ascending order. 
  % We know that $a<x_1$ and 
  % $x_n < b$, so we can 
  Define the new set $A^\prime$ obtained
  after deleting all the points as the union of open intervals
  $(a, x_1) \cup (x_2, x_3) \cup \ldots \cup (x_n, b)$. 
  The union of open intervals is open, thus $A^\prime$ is open.

  The same holds if a countable number of points are removed. 
  Follow the same procedure. Open sets remain open under arbitrary unions. 
  % Let $A=\mathbb{R}$ and remove $\mathbb{N}$ from $A$. 
  % Now $A^\prime$ is the union $(-\infty, 1) \cup (1, 2) \cup (2, 3) \ldots$, 
  % which is an open set. 
\item[]
\item[3.2.3.2]
  The sequence has strictly more elements than the set. 
  So if the set has an infinite number of terms approximating a limit, 
  then for sure the sequence will have those same terms.

  If no point of $A$ occurs more than finitely many times, 
  then you can take the sequence of points approximating the limit and
  deduplicate it to obtain a set of infinitely many points 
  approximating the limit. 
\item[]
\item[3.2.3.3]
  Every point of the set is now a limit. Every neighborhood of a point of $x$
  contains at least $x$; that's a limit under this new definition.
\item[]
\item[3.2.3.4]
  This seems like it follows trivially from the definitions. 
\item[]
\item[3.2.3.5]
  If $A$ is a closed set, $x$ is a point in $A$, and $B = A - \{ x \}$, 
  then $B$ is closed when $x$ is not a limit point of $A$. Why?
  Because $B$ will then contain all its limit points, just like $A$
  did, making it a closed set. 
\item[]
\item[3.2.3.6]
  
\item[]
\item[3.2.3.7]
  Every point of every open set is a limit point. 
  There are infinite number of points in each neighborhood of each point.
\item[]
\item[3.2.3.8]
\item[]
\item[3.2.3.9]
\item[]
\item[3.2.3.10]
\item[]
\item[3.2.3.11]
\item[]
\item[3.2.3.12]
\item[]
\item[3.2.3.13]
\item[]
\item[3.2.3.14]
  The empty set and the universe of all sets are both open and closed 
  by definition.
\item[]
\item[3.2.3.15]
  Let $\bigcup \mathcal{A}$ be a union of open intervals. 
  Replace every pair of intersecting intervals $(a, b), (c, d) \in \mathcal{A}$
  by the interval $(min(a, c), max(b, d))$. 
  % What's left is to prove that this is really equivalent. % TODO
\item[]
\item[3.2.3.16]
  Let $\bigcup \mathcal{A}$ and $\bigcup \mathcal{B}$ be two ways of
  describing the same open set. Let $(a_1, a_2)$ and $(b_1, b_2)$
  be the intervals in 
  $\mathcal{A}$ and $\mathcal{B}$, respectively,
  with the smallest first element. 
  Now $a_1 \le b_1$, otherwise $(b_1, b_2)$ would include the limit $a_1$.
  Likewise $b_1 \le a_1$ because 
  otherwise $(a_1, a_2)$ would include $b_1$. Hence $a_1 = b_1$. 
  A similar argument holds for $a_2$ and $b_2$, so the intervals 
  must be the same. Removing these smallest intervals and applying the
  same argument inductively, we see that $\mathcal{A}$ must be the 
  same set as $\mathcal{B}$.
\item[]
\end{enumerate}
