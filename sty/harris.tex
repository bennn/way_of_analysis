% ***********************************************************
% ******************* PHYSICS HEADER ************************
% ***********************************************************
% Original:
% http://www.dfcd.net/articles/latex/latex.html
% With additions by Harris Karsch

\documentclass[11pt]{article} 

% Basic things
\usepackage[latin1]{inputenc}
\usepackage{amsmath} % AMS Math Package
\usepackage{amsthm} % Theorem Formatting
\usepackage{amssymb}	% Math symbols such as \mathbb
\usepackage{amsfonts}
\usepackage{changepage}
\usepackage{bm}
\usepackage[shortlabels]{enumitem}
\usepackage{graphicx} % Allows for eps images
\usepackage{multicol} % Allows for multiple columns
\usepackage[dvips,letterpaper,margin=0.75in,bottom=0.5in]{geometry}
 % Sets margins and page size

% Extra packages, not installed on all computers
% comment/uncomment as necessary

\usepackage{tikz}
%\usetikzlibrary{calc}
\usepackage{calc}
\usepackage{cancel}
%\usepackage{marvosym}
\usepackage{pgfplots}
\pgfplotsset{compat=1.8}
\usepackage{calligra}

% Page setup
\pagestyle{empty} % Removes page numbers
\makeatletter % Need for anything that contains an @ command 
\renewcommand{\maketitle} % Redefine maketitle to conserve space
{ \begingroup \vskip 10pt \begin{center} \large {\bf \@title}
	\vskip 10pt \large \@author \hskip 20pt \@date \end{center}
  \vskip 10pt \endgroup \setcounter{footnote}{0} }
\makeatother % End of region containing @ commands
\renewcommand{\labelenumi}{(\alph{enumi})} % Use letters for enumerate

% Some not math but still useful commands
\newcommand{\s}{\textrm{ }}
\newcommand{\tab}{\hspace{2em}}
\newcommand{\swapcommands}[2]{\let\temp #1 \let #1 #2 \let #2 \temp}

% Useful math things
% \DeclareMathOperator{\Sample}{Sample}

% General Math
\newcommand{\abs}[1]{\left| #1 \right|} % for absolute value
\newcommand{\avg}[1]{\left< #1 \right>} % for average
\let\baraccent=\= % rename builtin command \= to \baraccent
\renewcommand{\=}[1]{\stackrel{#1}{=}} % for putting numbers above =
\newcommand{\ev}[1]{\left\langle #1 \right\rangle} % expected value
% same as \avg, yes, but I like different names for different contexts
\newcommand{\units}[1]{\left[#1\right]}
\newcommand{\set}[1]{\left\{#1\right\}}
\newcommand{\ip}[2]{\left\langle #1, #2 \right\rangle}
% Note to self - changed^, so might break some old psets.
\newcommand{\tr}{\textrm{tr}}
\newcommand{\re}{\textrm{Re}}
\newcommand{\im}{\textrm{Im}}
\newcommand{\rep}[1]{\re\left\{#1\right\}}
\newcommand{\imp}[1]{\im\left\{#1\right\}}
\newcommand{\sech}{\,\mathrm{sech}} % apparently doesn't exist already?
\newcommand{\wt}{\w t} % I don't like the space all the time
\newcommand{\II}{I\!I}
\newcommand{\oh}{\frac{1}{2}}
\newcommand{\lb}[2]{\left[ #1, #2\right]} % Lie Bracket

% Vectors
\let\vaccent=\v % rename builtin command \v{} to \vaccent{}
\renewcommand{\v}[1]{\ensuremath{\mathbf{#1}}} % for vectors
\newcommand{\gv}[1]{\ensuremath{\mbox{\boldmath$ #1 $}}} 
% for vectors of Greek letters
\newcommand{\uv}[1]{\ensuremath{\mathbf{\hat{#1}}}} % for unit vector
\newcommand{\cv}[1]{\tilde{\v{#1}}} % complex vector
\newcommand{\guv}[1]{\uv{\gv{#1}}}
\newcommand{\grad}[1]{\gv{\nabla} #1} % for gradient
\let\divsymb=\div % rename builtin command \div to \divsymb
\renewcommand{\div}[1]{\gv{\nabla} \cdot #1} % for divergence
\newcommand{\curl}[1]{\gv{\nabla} \times #1} % for curl
\newcommand{\del}{\nabla}
\newcommand{\ihat}{\boldsymbol{\hat{\textbf{\i}}}}
\newcommand{\jhat}{\boldsymbol{\hat{\textbf{\j}}}}
\newcommand{\cross}[2]{\left(#1_2 #2_3 - #1_3 #2_2\right)\v{i} -
  \left(#1_1 #2_3 - #1_3 #2_1\right)\v{j} +
  \left(#1_1#2_2 - #1_2 #2_1\right)\v{k}}

% Calculus
\let\underdot=\d % rename builtin command \d{} to \underdot{}
\renewcommand{\d}[2]{\frac{d #1}{d #2}} % for derivatives
\newcommand{\dd}[2]{\frac{d^2 #1}{d #2^2}} % for double derivatives
\newcommand{\pd}[2]{\frac{\partial #1}{\partial #2}} 
% for partial derivatives
\newcommand{\pdd}[2]{\frac{\partial^2 #1}{\partial #2^2}} 
% for double partial derivatives
\newcommand{\pdc}[3]{\left( \frac{\partial #1}{\partial #2}
 \right)_{#3}} % for thermodynamic partial derivatives
\newcommand{\intint}{\int\!\!\!\int}
% To put boundaries on both integrals, do it manually
\newcommand{\dx}{dx\,} % sometimes looks better with extra room
\newcommand{\intall}{\int_{-\infty}^{\infty}}

% Names of groups (because I like putting them in math environments)
\newcommand{\SU}{\textrm{SU}}
\newcommand{\Sp}{\textrm{Sp}}
\newcommand{\U}{\textrm{U}}
\newcommand{\SO}{\textrm{SO}}
\newcommand{\GL}{\textrm{GL}}
\newcommand{\SL}{\textrm{SL}}

% Quantum (i.e. Dirac notation)
\newcommand{\ket}[1]{\left| #1 \right>} % for Dirac bras
\newcommand{\bra}[1]{\left< #1 \right|} % for Dirac kets
\newcommand{\braket}[2]{\left< #1 \vphantom{#2} \right|
 \left.\! #2 \vphantom{#1} \right>} % for Dirac brackets
\newcommand{\matrixel}[3]{\left< #1 \vphantom{#2#3} \right|
 #2 \left| #3 \vphantom{#1#2} \right>} % for Dirac matrix elements
\newcommand{\evon}[2]{\bra{#2} #1 \ket{#2}}
\newcommand{\h}{\hbar}
\newcommand{\hoh}{\frac{\hbar}{2}}

% E&M things
\newcommand{\curlyr}{\textcalligra{r}} % not the real Griffiths one, but close
\newcommand{\Griffiths}{\textit{Griffiths }}
\newcommand{\Staelin}{\textit{Staelin}}
\newcommand{\emf}{\mathcal{E}}
\newcommand{\w}{\omega}
\newcommand{\eo}{\epsilon_0}
\newcommand{\muo}{\mu_0}
\newcommand{\e}{\epsilon}

% Proof stuff
\newtheorem{prop}{Proposition}
\newtheorem{thm}{Theorem}[section]
\newtheorem*{thm*}{Theorem}
\newtheorem{lem}[thm]{Lemma}
\theoremstyle{definition}
\newtheorem{dfn}{Definition}
\newtheorem*{dfn*}{Definition}
\theoremstyle{remark}
\newtheorem*{rmk}{Remark}
\newcommand{\st}{\textrm{ s.t. }}


% ***********************************************************
% ********************** END HEADER *************************
% ***********************************************************
